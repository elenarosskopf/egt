%----------------------------------------------------------------------------------------
%	PACKAGES AND OTHER DOCUMENT CONFIGURATIONS
%----------------------------------------------------------------------------------------

\documentclass[DIV=calc, paper=a4, fontsize=11pt, twocolumn]{scrartcl}	 % A4 paper and 11pt font size

\usepackage[english]{babel} % English language/hyphenation
\usepackage[protrusion=true,expansion=true]{microtype} % Better typography
\usepackage{amsmath,amsfonts,amsthm} % Math packages
\usepackage[svgnames]{xcolor} % Enabling colors by their 'svgnames'
\usepackage[hang, small,labelfont=bf,up,textfont=it,up]{caption} % Custom captions under/above floats in tables or figures
\usepackage{booktabs} % Horizontal rules in tables
\usepackage{fix-cm}	 % Custom font sizes - used for the initial letter in the document

\usepackage{sectsty} % Enables custom section titles
\allsectionsfont{\usefont{OT1}{phv}{b}{n}} % Change the font of all section commands

\usepackage{fancyhdr} % Needed to define custom headers/footers
\pagestyle{fancy} % Enables the custom headers/footers
\usepackage{lastpage} % Used to determine the number of pages in the document (for "Page X of Total")
\usepackage{hyperref}

% Headers - all currently empty
\lhead{}
\chead{}
\rhead{}

% Footers
\lfoot{}
\cfoot{}
\rfoot{\footnotesize Page \thepage\ of \pageref{LastPage}} % "Page 1 of 2"

\renewcommand{\headrulewidth}{0.0pt} % No header rule
\renewcommand{\footrulewidth}{0.4pt} % Thin footer rule

\usepackage{lettrine} % Package to accentuate the first letter of the text
\newcommand{\initial}[1]{ % Defines the command and style for the first letter
\lettrine[lines=3,lhang=0.3,nindent=0em]{
\color{DarkGoldenrod}
{\textsf{#1}}}{}}

%----------------------------------------------------------------------------------------
%	TITLE SECTION
%----------------------------------------------------------------------------------------

\usepackage{titling} % Allows custom title configuration

\newcommand{\HorRule}{\color{DarkGoldenrod} \rule{\linewidth}{1pt}} % Defines the gold horizontal rule around the title

\pretitle{\vspace{-30pt} \begin{flushleft} \HorRule \fontsize{50}{50} \usefont{OT1}{phv}{b}{n} \color{DarkRed} \selectfont} % Horizontal rule before the title

\title{Evolutionary Game Theory} % Your article title

\posttitle{\par\end{flushleft}\vskip 0.5em} % Whitespace under the title

\preauthor{\begin{flushleft}\large \lineskip 0.5em \usefont{OT1}{phv}{b}{sl} \color{DarkRed}} % Author font configuration

\author{Elena Rosskopf and Ellen Halaburt, } % Your name

\postauthor{\footnotesize \usefont{OT1}{phv}{m}{sl} \color{Black} % Configuration for the institution name
Module Evolutionary Game Theory WS 2014/15 % Your institution

\par\end{flushleft}\HorRule} % Horizontal rule after the title

\date{\today} % Add a date here if you would like one to appear underneath the title block

%----------------------------------------------------------------------------------------

\begin{document}

\maketitle % Print the title

\thispagestyle{fancy} % Enabling the custom headers/footers for the first page 

%----------------------------------------------------------------------------------------
%	ABSTRACT
%----------------------------------------------------------------------------------------

% The first character should be within \initial{}
\initial{H}\textbf{ere is our super nice abstract with the solution which will bring peace to the world}

%----------------------------------------------------------------------------------------
%	ARTICLE CONTENTS
%----------------------------------------------------------------------------------------

\section*{Explanation of problem}

\noindent why is it important to such things \
modelling of nature for forecasting, understanding, estimating  \
get old paradigm of the behaviour in HD right - paradigm shift to the new paradigm that HD players could act in a way not known until now


\begin{align}
A = 
\begin{bmatrix}
A_{11} & A_{21} \\
A_{21} & A_{22}
\end{bmatrix}
\end{align}

Looking at evolutionary game structures, the pay-off is crucial to determine each players fitness and the fitness of a whole group. 


%------------------------------------------------

\subsection*{Prisoners Dilemma}

Explanation of PD in general:
The cooperators get exploited by defectors, subsequently defectors are naturally selected. The cost to the donor of fitness(pay-off?) is always higher than zero, but generally lower the benefit to the receiver of the pay-off ( b > c > 0 ). The defector‘s pay-off is the highest pay-off b if the other player is cooperating. The lowest pay-off, namely only the cost, has then (-c) the cooperator which is defected in the unilateral cooperation. Finally it is best to defect regardless of other players decision. Mutual defections result then in pay-off zero for both players, not reducing the fitness but also not increasing it (see \ref{table1}). Here, the defector strategy is the ESS. 

\begin{table}[h]
\caption{Prisoner's Dilemma}
\label{table1}
\centering
\begin{tabular}{lll}
%\toprule
%\multicolumn{2}{c}{Name} \\
%\cmidrule(r){1-2}
 & C & D \\
\midrule
Payoff to C & $b-c$ & $-c$ \\
Payoff to D & $b$ & $0$ \\
\bottomrule
\end{tabular}
\end{table}


%------------------------------------------------

\subsection*{Snowdrift game}

In general: \\
At the snowdrift game we have the difference to Prisoners Dilemma. The players can share the benefit and the cost, depending on their strategies. Another feature in this game is, that if one defects, the pay-off could be less than the sucker's pay-off of the unilateral cooperator. Still defecting if the other player cooperates is, according to the matrix, the best response choice. As one can see in table \ref{table2}, the pay-off matrix is slightly different than the one of Prisoners Dilemma in table \ref{table1}. 
If $2b > c > b > 0$, meaning that if costs are high, these pay-off structures change the game to a PD and affect the reverse pay-off structure. If $b > c > 0$ , the best action depends on co-players action resulting in a mixed strategy population, where rare strategies can invade, either defector or cooperator with an ESS at cooperator proportion is $1- c/(2b-c)$. 


\begin{table}[h]
\caption{Snowdrift game}
\label{table2}
\centering
\begin{tabular}{lll}
%\toprule
%\multicolumn{2}{c}{Name} \\
%\cmidrule(r){1-2}
 & C & D \\
\midrule
Payoff to C & $b-c/2$ & $b-c$ \\
Payoff to D & $b$ & $0$ \\
\bottomrule
\end{tabular}
\end{table}

%------------------------------------------------

\section*{task 1}


\begin{description}
\item[First] explanation of spatial modelling structure
\item[Last] explanation of experiments
\end{description}

\section*{task 2}
different plots like in the paper from hauert with the frequency of coop against cb ratio \
explanation of differences \
explanation of HD special case


\section*{task 3}

\section*{discussion}

maybe discuss difference to the hauert paper? 
differ our results with their results? 

where can we give in more time, where are options to prolong this topic? relation to nature, where is the importance here ? 

bibiography: this paper should be enough though :D 

\nocite{Albizu2013}
%----------------------------------------------------------------------------------------
%	REFERENCE LIST
%----------------------------------------------------------------------------------------
\bibliography{literatur}
\bibliographystyle{plain}

%----------------------------------------------------------------------------------------

\end{document}
