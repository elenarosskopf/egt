%----------------------------------------------------------------------------------------
%	PACKAGES AND OTHER DOCUMENT CONFIGURATIONS
%----------------------------------------------------------------------------------------

\documentclass[DIV=calc, paper=a4, fontsize=11pt, twocolumn]{scrartcl}	 % A4 paper and 11pt font size

\usepackage[english]{babel} % English language/hyphenation
\usepackage[protrusion=true,expansion=true]{microtype} % Better typography
\usepackage{amsmath,amsfonts,amsthm} % Math packages
\usepackage[svgnames]{xcolor} % Enabling colors by their 'svgnames'
\usepackage[hang, small,labelfont=bf,up,textfont=it,up]{caption} % Custom captions under/above floats in tables or figures
\usepackage{booktabs} % Horizontal rules in tables
\usepackage{fix-cm}	 % Custom font sizes - used for the initial letter in the document

\usepackage{sectsty} % Enables custom section titles
\allsectionsfont{\usefont{OT1}{phv}{b}{n}} % Change the font of all section commands

\usepackage{fancyhdr} % Needed to define custom headers/footers
\pagestyle{fancy} % Enables the custom headers/footers
\usepackage{lastpage} % Used to determine the number of pages in the document (for "Page X of Total")

% Headers - all currently empty
\lhead{}
\chead{}
\rhead{}

% Footers
\lfoot{}
\cfoot{}
\rfoot{\footnotesize Page \thepage\ of \pageref{LastPage}} % "Page 1 of 2"

\renewcommand{\headrulewidth}{0.0pt} % No header rule
\renewcommand{\footrulewidth}{0.4pt} % Thin footer rule

\usepackage{lettrine} % Package to accentuate the first letter of the text
\newcommand{\initial}[1]{ % Defines the command and style for the first letter
\lettrine[lines=3,lhang=0.3,nindent=0em]{
\color{DarkGoldenrod}
{\textsf{#1}}}{}}

%----------------------------------------------------------------------------------------
%	TITLE SECTION
%----------------------------------------------------------------------------------------

\usepackage{titling} % Allows custom title configuration

\newcommand{\HorRule}{\color{DarkGoldenrod} \rule{\linewidth}{1pt}} % Defines the gold horizontal rule around the title

\pretitle{\vspace{-30pt} \begin{flushleft} \HorRule \fontsize{50}{50} \usefont{OT1}{phv}{b}{n} \color{DarkRed} \selectfont} % Horizontal rule before the title

\title{Evolutionary Game Theory} % Your article title

\posttitle{\par\end{flushleft}\vskip 0.5em} % Whitespace under the title

\preauthor{\begin{flushleft}\large \lineskip 0.5em \usefont{OT1}{phv}{b}{sl} \color{DarkRed}} % Author font configuration

\author{Elena Rosskopf and Ellen Halaburt, } % Your name

\postauthor{\footnotesize \usefont{OT1}{phv}{m}{sl} \color{Black} % Configuration for the institution name
Module Evolutionary Game Theory WS 2014/15 % Your institution

\par\end{flushleft}\HorRule} % Horizontal rule after the title

\date{\today} % Add a date here if you would like one to appear underneath the title block

%----------------------------------------------------------------------------------------

\begin{document}

\maketitle % Print the title

\thispagestyle{fancy} % Enabling the custom headers/footers for the first page 

%----------------------------------------------------------------------------------------
%	ABSTRACT
%----------------------------------------------------------------------------------------

% The first character should be within \initial{}
\initial{H}\textbf{ere is some sample text to show the initial in the introductory paragraph of this template article. The color and lineheight of the initial can be modified in the preamble of this document.}

%----------------------------------------------------------------------------------------
%	ARTICLE CONTENTS
%----------------------------------------------------------------------------------------

\section*{The Algebra of the Nash Equilibrium}

The simulation for the third task ended in a Nash Equilibrium
for the snow drift game. In order to verify the result of our
simulation we will derive the probability for the mixed strategy
for the ``defector player''.


The payoff matrix for the HD game is:

\begin{tabular}{l|ll}
  & Left & Right \\
\midrule
Up & $b-(c/2)$ & $b-c$ \\
Down & $b$ & $0$ \\
\end{tabular}

We simulated the game for benefit $b=1$ and cost $c=0.75$.
So the actual payoff matrix is:

\begin{tabular}{l|ll}
  & Left & Right \\
\midrule
Up & $0.625$ & $0.25$ \\
Down & $1$ & $0$ \\
\end{tabular}

Let $\sigma$ be the probability for the mixed strategy
to play $C$ and let $U_x$ be the utility function for the
player when playing strategy ``Left'' ($L$) or ``Right'' ($R$)
respectively.

We get the equation system:
$$\begin{align*}
  U_L &= U_R\\
  U_L &= f(\sigma)\\
  U_R &= f(\sigma)
\end{align*}$$

We can further define (this is just the probability
of the strategy multiplied with the payoff, for both
strategies):
$$\begin{align*}
  U_L &= 0.625\sigma + 1 \cdot (1-\sigma)\\
  U_R &= 0.25\sigma + 0 \cdot (1-\sigma)
\end{align*}$$

Set the two equations for $U_L = U_R$, we get:

$$  0.625\sigma + (1-\sigma) =  0.25\sigma$$

Solving the equation is left as an exercise to the reader ;-)

As a hint:

$$\sigma = 0.60$$

%------------------------------------------------

\section*{Section 2}


\begin{description}
\item[First] This is the first item
\item[Last] This is the last item
\end{description}

\nocite{Albizu2013}
%----------------------------------------------------------------------------------------
%	REFERENCE LIST
%----------------------------------------------------------------------------------------
\bibliography{literatur}
\bibliographystyle{plain}

%----------------------------------------------------------------------------------------

\end{document}
